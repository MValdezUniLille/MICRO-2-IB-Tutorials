\documentclass{article}
\usepackage[english]{babel}
\usepackage{textcmds}
\usepackage{csquotes}
\usepackage{comment}
\usepackage{float}
\usepackage{enumitem}
\usepackage{hyperref}
\usepackage{mathtools}
\usepackage[style=authoryear,
backend=biber,
giveninits=true,
uniquelist = false,
uniquename=init,
isbn=false,
maxcitenames=2,
dashed=false,
maxbibnames=999,
doi=false,
url=false]{biblatex}
\addbibresource{bibliography.bib}
%--------------------------------------------------------------
%
% HERE IS WHERE THE DOCUMENT STARTS!
%
%--------------------------------------------------------------
\begin{document}
\subsection*{Homework Assignment 1}
(Based of material from Chapters 1 to 16 of \textcite{Valrian_Inter_Micro} reviewed during the first three weeks of the course)
\begin{enumerate}
  \item Show the effect of the emergence of AirBnB as an alternative to hotel accommodation on
  \begin{enumerate}
    \item The residential housing market
    \item On the market for hotel rooms
  \end{enumerate}
  \item Research in China showed that when poor households receive an opportunity to purchase rice at subsidized prices (below market prices), the amount of rice that they consume actually falls.
  \begin{enumerate}
    \item Explain why this might be the case.
    \item Show graphically the kind of indifference curves that could lead to this kind of outcome.
  \end{enumerate}
  \item Why do taxi drivers work fewer hours when demand is high?
  \item Reproduce Figure 7.4 from the book and
  \begin{enumerate}
    \item Draw the two indifference curves that would result in the consumer optimally choosing the bundles (x1, x2) and (y1, y2).
    \item Show that these two indifference curves are not consistent with each other.
    \end{enumerate}
  \item During the Dutch War of Independence (also known as the Eighty Years War, 1568-1648), the United Provinces of the Netherlands successfully waged war against the much larger Spanish Empire. Their success in this war was attributed, among other factors, to the ability of the Dutch Treasury to rely on a steady flow of revenue from excise taxes on beer. At that time, beer was an important part of Northern European diet, providing nutrition and being safer to drink than water. In contrast, the Spanish military campaigns were financed mainly by imports of gold and silver from its colonies in the New World.
  \begin{enumerate}
    \item Show how the fact that beer was a staple part of diet at the time allowed the Dutch Treasury to raise tax revenue without causing much deadweight loss or hurting the beer producers.
  \end{enumerate}

\end{enumerate}

\printbibliography

\end{document}
