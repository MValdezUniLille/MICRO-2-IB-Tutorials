\documentclass{article}
\usepackage[english]{babel}
\usepackage{textcmds}
\usepackage{csquotes}
\usepackage{comment}
\usepackage{float}
\usepackage{enumitem}
\usepackage{hyperref}
\usepackage{mathtools}
\usepackage[style=authoryear,
backend=biber,
giveninits=true,
uniquelist = false,
uniquename=init,
isbn=false,
maxcitenames=2,
dashed=false,
maxbibnames=999,
doi=false,
url=false]{biblatex}
\addbibresource{bibliography.bib}
%--------------------------------------------------------------
%
% HERE IS WHERE THE DOCUMENT STARTS!
%
%--------------------------------------------------------------
\begin{document}
\subsection*{Homework Assignment 2}
\begin{enumerate}
  \item A firm is using two inputs, \(x_1\) and \(x_2\), to produce one output, \(y\). The input prices are \(w_1\) and \(w_2\), respectively. The price of \(x_1\) (\(w_1\)) has increased, while \(w_2\) has remained unchanged. The firm has responded to this by changing how much of each input it uses without altering the quantity of its output.
  \begin{enumerate}
    \item What has happened to the firm’s use of \(x_1\) and \(x_2\)? Show your answer graphically.
    \item Have the overall costs of the firm gone up or down?
  \end{enumerate}
  \item Show who bears the burden of an excise tax in the short run and long run. Is the tax burden borne by the consumer or the producer, or by both?
  \item Taxi licenses and public transport: discuss the implications of expanding public transport network on the cost of taxi licenses.
  \item Patents offer inventors a temporary monopoly and thus allow then to enjoy higher profits than in a competitive environment while the patent lasts. This allows them to sell their products well above the marginal cost. It has been argued that in the case of life-saving medications (for example antiviral drugs against AIDS), the patent protection should be short-lived or none so that the medicines can be distributed cheaply, especially in less developed countries. Discuss.
\end{enumerate}
\nocite{Valrian_Inter_Micro}

\newpage

\textbf{Answers}


\begin{enumerate}
  \item The slope of the isocost lines changes in response to the change in w1. Output staying the same implies that the firm operates on the same isoquant as before. Because of the price change, the firm will use less \(x_1\) than before, and more \(x_2\). Since the firm is on the same isoquant as before, it has to move to a higher isocost line so that the overall cost increases: the original isocost line becomes steeper and no longer is tangent to the isoquant associated with \(y\).
  \item In an industry with free entry, a tax will initially raise the price to the consumers by less than the amount of the tax, since some of the incidence of the tax will fall on the producers. But in the long run the tax will induce firms to exit from the industry, thereby reducing supply, so that consumers will eventually end up paying the entire burden of the tax. See Fig 24.6.
  \item Taxi licenses impede entry into the industry. As a result, the cost of the taxi license is inflated, and taxi drivers operate with zero profits. Expanding the public transport network makes taking taxis less attractive. This should lead to a fall in the cost of the taxi license. Taxi drivers still operate with zero profits, but the licenses that they paid for have become less valuable.
  \item The monopoly due to the patent causes a deadweight loss of monopoly, with the resulting prices higher and quantity produced lower than under competitive equilibrium. However, this allows the companies to recoup the cost of research and development. Without patents, few companies would be willing to accept substantial up-front costs associated with R\&D. See Example on p. 467.
\end{enumerate}


\printbibliography
\end{document}
