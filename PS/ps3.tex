\documentclass{article}
\usepackage[english]{babel}
\usepackage{textcmds}
\usepackage{csquotes}
\usepackage{comment}
\usepackage{float}
\usepackage{enumitem}
\usepackage{hyperref}
\usepackage{mathtools}
\usepackage[style=authoryear,
backend=biber,
giveninits=true,
uniquelist = false,
uniquename=init,
isbn=false,
maxcitenames=2,
dashed=false,
maxbibnames=999,
doi=false,
url=false]{biblatex}
\addbibresource{bibliography.bib}
%--------------------------------------------------------------
%
% HERE IS WHERE THE DOCUMENT STARTS!
%
%--------------------------------------------------------------
\begin{document}

This problem set is based on Chapter 15 \& 16 of \cite{Valrian_Inter_Micro}

\subsubsection*{Problem 6: Market Demand and Elasticity}
Suppose the fisherman we talked about last time faces the following inverse
demand curve in the local market:
\begin{equation*}
  P(q) = a-bq
\end{equation*}
\begin{itemize}
  \item What is the demand curve he faces? Solve for it.
  \item Derive a close form solution for the price elasticity of demand.
  \item At which point is the price elasticity of demand \(0, -1\) and \(-\infty\)
\end{itemize}
Now suppose that the inverse demand function he faces is instead given by:
\begin{equation*}
  p(q) = \frac{1}{A}q^{\frac{1}{-\epsilon}}
\end{equation*}
\begin{itemize}
  \item What is the price elasticity of demand?
  \item What is the fisherman's \textit{revenue} \(R\), as a function of price \(p\)?
  \item What is the relationship between the fisherman's revenue and the price
  elasticity of demand in the local market?
  \item What is the key parameter, \(A\) or \(\epsilon\)?
  \item For which values of the key parameter you just identified does the fisherman's
  revenue increases, decreases or stays the same if he increases the price at which
  he sells his fish?
\end{itemize}

 \newpage
\subsubsection*{Problem 7: Market Equilibrium}
Let us follow the exercise in the book. Suppose that the local taco market has
demand function of \(Q_D(p) = a-bp\) and a supply funtion of \(Q_S = dp - c\)
\begin{itemize}
  \item Graph both curves.
  \item What is the market equilibrium? Derive the close-form expression.
  \item Now suppose a local politician implements a per-taco tax. What happens
  to the local taco market? Is there deadweight loss? Who pays for the tax,
  taqueros (suppliers) or taco-eaters (consumers)?
  \item Now imagine that, instead of analyzing the taco market, we are analyzing
  the market for insulin. A representative person in the insulin market requires
  at least \(Q_{ins}\) per day to stay alive, not more, not less. Suppose that
  insulin suppliers share the same supply curve as taqueros. What can you say
  about insulin's price elasticity of demand?
  \item If a local politician decided to implement a tax on insulin, who would bear
  the tax? Would there be a deadweight loss? Discuss.
  \item Now imagine that, instead of analyzing the insulin market, we are analyzing
  the market for a specific type of tree that takes thousands of years to grow.
  Suppose that demand for this tree is \(Q_D(p) = a-bp\), and that the amount of
  tress ready to harvest is \(Q_{trees}\). If a local politician decided to implement
  a per-tree tax, who would bear the tax?
\end{itemize}
\newpage
\printbibliography

\end{document}
