\documentclass{article}
\usepackage[english]{babel}
\usepackage{textcmds}
\usepackage{csquotes}
\usepackage{comment}
\usepackage{float}
\usepackage{enumitem}
\usepackage{hyperref}
\usepackage{mathtools}
\usepackage[style=authoryear,
backend=biber,
giveninits=true,
uniquelist = false,
uniquename=init,
isbn=false,
maxcitenames=2,
dashed=false,
maxbibnames=999,
doi=false,
url=false]{biblatex}
\addbibresource{bibliography.bib}
%--------------------------------------------------------------
%
% HERE IS WHERE THE DOCUMENT STARTS!
%
%--------------------------------------------------------------
\begin{document}

This problem set is based on  \cite{Valrian_Inter_Micro}

\subsubsection*{Problem 8: Technology, Costs and Profit Maximization }
Let us now inmerse ourselves in the taco market. We focus on the supply-side.
Suppose that there are enough taquerias to support a competitive market,
and suppose that they all use the same technology, captured by a
Cobb-Douglas production function that takes as input the amount of
available \textit{capital equipment \qq{k}} and amount of \textit{labor \qq{l}}
and outputs tacos produced.
We will make use of the idea of a \textit{representative taqueria}.
The production function is:
\begin{equation*}
  y = f(k,l) = k^{\alpha}l^{1-\alpha}
\end{equation*}
Where \(\alpha \in (0,1)\).
\begin{itemize}
  \item Fix output at \(\tilde{y}\). Draw the isoquant graph.
  (Hint: on \((k,l)-space\)).
  \item What is the marginal product of capital? And of labor?
  \item What type of production function is this?
  (Constant returns to scale, decreasing, increasing?), why?
  \item What is the technical rate of substitution?
  \item Suppose that \(r, w\) are the prices of 1 unit of capital and labor,
  respectively, while a single taco sells at price \(p\).
  Set up the profit maximization problem of the firm, but do not solve it yet.
  What are the endogenous variables? i.e., what are we solving for?
  How realistic is this model of firm behaviour? Discuss.
  \item Suppose total costs are equal to \(\bar{c}\). Draw the isocost line.
  \item Suppose that capital is fixed at \(\tilde{k}\). How is the firm's profit
  maximization problem affected? What are we now solving for?
  \item Solve for the optimal labor demand. What is the optimality condition that
  must be satisfied for profits to reach a maximum?
  \item Do some comparative statics: What happens to output
  if \(p\) moves? If \(w\) moves? If \(r\) moves?
  \item Now suppose that the taqueria is deciding how much capital to acquire in
  order to maximize profits. How does this change the taqueria's problem?
  \item Try to find the optimal factor allocation. What do you notice? What can
  you do about it?
\end{itemize}
\newpage

\printbibliography

\end{document}
